\chapter{Source Code}
\label{app:sourcecode}
\lhead{Appendix \ref{app:sourcecode}. \emph{Source Code}}

The complete project source code is available for download online, due to its significant size. Both the robot firmware and the embedded Bluetooth stack may be viewed and modified according to the licence agreements included at the top of each source file in the download package.

\section{Obtaining the Source Code}

All relevant material relating to this project (including source code, schematics, and this document) may be obtained from the official project page, located at \\ \href{http://www.fourwalledcubicle.com/ExplorerBot.php}{\textit{http://www.fourwalledcubicle.com/ExplorerBot.php}}.

\section{Build Dependencies}

The Bluetooth stack and robot firmware was written in the C language, and targeted at the free open source AVR-GCC compiler and avr-libc library. A standard \texttt{makefile} included with the firmware allows for command line control over the building of the project files into a set of binaries which can then be programmed into the target microcontroller for use via the command \texttt{make all}. The following tools are required to build the firmware under Windows:

\begin{itemize}
	\item The \textbf{WinAVR 20100101} release download, or Windows binaries of the \textbf{GNU Shell Utilities}
	\item The latest \textbf{AVR Toolchain} release download from Atmel (included with Atmel's free \textit{AVRStudio 5} software)
\end{itemize}

Under Debian Linux environments, the following packages are required:

\begin{itemize}
	\item \textbf{gcc-avr} 
	\item \textbf{binutils-avr}
	\item \textbf{avr-libc}
	\item \textbf{avrdude}
\end{itemize}

Which can be installed via the command prompt using the command \texttt{sudo apt-get install gcc-avr binutils-avr avr-libc avrdude}.