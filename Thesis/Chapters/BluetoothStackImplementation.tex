\chapter{Bluetooth Stack Implementation}
\label{chp:btstackimp}
\lhead{Chapter \ref{chp:btstackimp}. \emph{Bluetooth Stack Implementation}}

The development of the Bluetooth stack followed a bottom-up development methodology, designing and writing each software layer in sequence from the lowest layer first, to the highest and most abstracted layers last. This sequence ensured that each layer was functionally correct and could be verified before the higher layers were implemented. This suited the development of the Bluetooth stack---and, indeed, most software stacks---as each higher software layer is dependant solely on lower logical layers in the stack.

\section{Software Overview}

The software layers implemented in the completed Bluetooth stack are shown in Figure \ref{fig:completedbtstacklayers}.

\begin{figure}[tbph]
	\vspace{1em}
	\centering
		\includegraphics[width=100mm]{CompletedBluetoothStack.png}
	\rule{35em}{0.5pt}
	\caption[Diagram of the completed Bluetooth stack layers]{Diagram showing the completed layers of the Bluetooth stack.}
	\label{fig:completedbtstacklayers}
\end{figure}

Each software layer was implemented as a seperate pair of source code module files, written in the C language and targeted towards the C99 language specification. Configuration for the stack is located in the file \texttt{BluetoothCommon.h}, which must be included in the user application to operate the stack.

By default, only the Bluetooth stack core (giving the HCI and L2CAP layers) is integrated; services must be linked into the stack manually via the provided event and callback routines (see later in this chapter) to provide service-level functionality on top of the base stack functionality. This includes the SDP server service; while generally a core requirement of most Bluetooth devices, it may be omitted in applications not exposing server role services to save space in the resulting application binary.

\section{Design Considerations}

A number of design restrictions were placed on the development of the Bluetooth stack; these ensured that the completed stack remained modular, extensible and suitable for integration in a wide range of embedded systems with various capabilities and, conversely, limitations. As an unfortunate but expected side effect, a number of these restrictions complicated the software development process significantly.

\FloatBarrier
\subsection{No Heap Memory Allocations}

The largest restriction placed on the stack design was the requirement of static and stack allocated memory allocation only --- no dynamic heap allocated memory was to be used via the usual \lstinline{malloc()} and \lstinline{free()} functions from the \texttt{libc} library. Most embedded environments contain three types of memory allocations:

\begin{enumerate}
	\item \textbf{Static Allocations}, including global variables and \lstinline{static} qualified variables.
	\item \textbf{Stack Allocations}, automatically populated to store local automatic variables, function paramaters, return addresses, etc.
	\item \textbf{Heap Allocations}, persistant dynamic allocations manually managed by the user application.
\end{enumerate}

However, for significantly memory constrained systems, heap allocations are generally avoided due to the possibility of memory fragmentation; large numbers of differently-sized allocations and deallocations may cause fragmented holes to appear within the heap environment, causing allocation exhaustion to occur even if sufficient (raw) memory is available for an allocation request. In addition, dynamic memory complicates the formal analysis of a program, making memory requirements harder to judge, and thus harder to determine whether a given set of firmware will operate correctly within a specified environment without further in-depth analysis. In large embedded environments (ones with virtual memory management, and/or a large heap space) these problems are less of an issue and dynamic memory may be preferable for its simplicity.

The design of embedded stacks based on a no-heap-allocation strategy is a complex one, requiring certain trade-offs to be made. In the case of the Bluetooth stack presented here, a major side-effect of this decision was the need to fix the maximum number of simultaneous HCI device connections, L2CAP channel connections, and other queue- and list-like data structures within each instance of the stack. A major downside to this approach is the increased complexity of packet buffering and re-assembly, as fixed-size buffers must be used.

\FloatBarrier
\subsection{Endianness Correction}

A second complication based on the physical constraints from the execution environment is the native endianness of the processor hardware, i.e., the order in which multiple-byte data values are stored and interpreted inside the processor registers. Depending on the execution architecture, data values may be represented in Big Endian (most significant byte at the lowest address) or Little Endian (least significant byte at the lowest address) format within the CPU. Data transmitted and received at the local device to remote device boundary must undergo a encoding conversion if the endianness of the source and sink do not match.

In the case of the Bluetooth specification, this was further complicated by the various layers; some layers used Little Endian encoding for exchanged data, while others selected a Big Endian format. To perform the correct encoding/decoding on each platform, a set of conversion macros were implemented at points where multi-byte values were exchanged. This ensured that regardless of the native endian format of the architecture the stack is compiled to, multi-byte values will be interpreted correctly.

\FloatBarrier
\subsection{Physical Transport Independance}

As the Bluetooth 2.1 specification outlines a several possible physical transports of HCI packets between the Bluetooth transceiver silicon and the application microcontroller, it was important that the physical transport layer be made independant from the rest of the stack. By making no assumptions about the form of transport protocol used to transfer packets to and from the attached Bluetooth silicon, the user application is free to implement their own transport mechanism without having to modify the internals of the stack. One instance of the stack may run across a USB connection to an attached Bluetooth module, while another may simultaneously connect to the microcontroller over a USART link.

\FloatBarrier
\subsection{Bluetooth Specification Compatibility}

For a software stack to be suitable for general use, it must be designed around, and undergo a series of tests to ensure compatibility with, a particular version of the protocol specification. Without a fixed version of a specification, correctness of the finished stack and compatibility with other implementations of the same technology cannot be assured.

As a \textit{compliant} stack would require extensive development and expensive certification procedures, the stack was instead aimed to be \textit{compatible} with version 2.1 of the Bluetooth specification. This subtle difference in the terminology indicates a large difference in the restrictions around the exact implementation. While a \textit{compliant} stack must be verified against the entire specification and rigorously tested to ensure exact conformance, a \textit{compatible} must only work with (most) other devices implementing their own compliant or compatible stacks of the same protocol version.

In the future the stack could be made conformant if desired, however for the purposes of the project only protocol compatibility was tested.

\FloatBarrier
\subsection{Multiple Stack Instances}

For maximum utility, the stack was designed to allow for multiple simultaneous stack instances. This was implemented to allow for unusual Bluetooth usage scenarios where more than one tranceiver is desirable in a single system, either for security, wireless signal coverage or bandwidth reasons. Each physical transceiver is allocated a logical stack instance, capable of sustaining one or more simultaneous connections, and one or more logical channels between devices. This flexibility ensures that the majority of usage scenarios are covered by the stack's implementation.

Code-wise, the multiple stack functionality is implemented as a common first parameter to all functions within the stack, \lstinline{BT_StackConfig_t* const StackState}. This parameter is used to pass around a reference to the stack instance being operated upon, and is somewhat analagous to the \lstinline{this} property in C++ classes, if the entire stack was wrapped in a single class. As the stack was written in the C language which lacks Object Orientated language capabilities, this form of basic polymorphism was implemented manually. For each physical tranceiver, the user application is expected to declare a new instance of the \lstinline{BT_StackConfig_t} structure within their code, and pass this to the main Bluetooth stack management routines (see Listing \ref{lst:stackconfig}).

\lstinputlisting[float=tbph,caption={Configuration example of a Bluetooth stack instance.},label={lst:stackconfig}]{./Figures/StackConfig.c}

\FloatBarrier
\subsection{Multiple Simultaneous Connections and Channels}

The HCI connections and L2CAP channels for each stack instance are implemented as an pair of object pools within the \lstinline{BT_StackConfig_t} structure; L2CAP channel objects within the pool are shared amongst all established HCI connections --- for example, one HCI connection may use all the available L2CAP channels, or two connections may share the pool equally. This design decision allows the full pool of L2CAP channel objects to be used by the stack regardless of the HCI connection requesting them. Figure \ref{fig:stackobjectpools} illustrates a single stack instance with a typical usage scenario, with two of the stack's HCI objects containing established connections, each with several L2CAP channel objects associated with the connection from the channel object pool.

\begin{figure}[tbph]
	\vspace{1em}
	\centering
		\includegraphics[width=140mm]{StackObjectPools.png}
	\rule{35em}{0.5pt}
	\caption[Diagram of the Bluetooth stack instance object pools]{Diagram showing the HCI Connection and L2CAP object pools within a single Bluetooth stack instance.}
	\label{fig:stackobjectpools}
\end{figure}

To positively link one used L2CAP channel instance to its parent HCI connection object, the HCI connection's \texttt{handle} property is stored in the L2CAP channel object as a primary key. This handle value, allocated uniquely by the external Bluetooth HCI controller silicon when a new HCI connection is made, is then used to locate the HCI connection object within a stack instance from an established channel instance. While a pointer to the parent HCI object would offer a faster method of locating the associated HCI connection, this would tightly couple the HCI layer with the L2CAP layer as the former would need to alter the latter's objects if and when HCI connections are terminated.

As the stack does not use heap-based dynamic memory allocation, this approach ensures that the stack contains similar levels of flexability to a dynamically allocated object approach, with only a minimal overhead. The maximum size of each object pool is set via the Bluetooth Stack's \lstinline{BT_MAX_DEVICE_CONNECTIONS} and \lstinline{BT_MAX_LOGICAL_CHANNELS} configuration defines.

\FloatBarrier
\subsection{Minimal Memory Usage}

While large embedded systems may run the completed Bluetooth stack, the design decisions made during its development were primarily aimed at ensuring the best performance and smallest footprint in severely resource-constrained microcontrollers. This resulted in the stack being optimised for small systems; only the minimal attributes required to establish and maintain connections are stored by the stack. One example of these optimizations is in the event system (described in detail later in this chapter) implementation: rather than implementing a callback registration system for each event, named callbacks are used instead. Various portions of the Bluetooth stack expect the user application to provide callback and event functions using names and prototypes defined in the stack header files.

This named callback function system reduces the memory footprint of the stack, as the callback addresses do not need to be stored in RAM at runtime, and the compiler does not need to inflate the binary with additional functions to manage the callback function registrations.

\section{Software Layer Implementation}

In this section, the design and implementation of each individual software layer that comprises the complete Bluetooth stack is described in detail.

\FloatBarrier
\subsection{Physical Transport}

While not part of the completed stack proper, a functional USB transport layer was completed for the project as an sample implementation in the user application, and hooked into the stack via the provided APIs. The Bluetooth USB transport requires the use of four logical USB communication channels (``pipes'' in USB terminology) to the USB Bluetooth adapter; the mandatory control pipe for HCI commands, a pair of \textit{Bulk} type pipes for data packets to and from the controller, and an \textit{Interrupt} pipe for HCI event notifications.

\begin{figure}[tbph]
	\vspace{1em}
	\centering
		\includegraphics[width=100mm]{BluetoothUSBTransport.png}
	\rule{35em}{0.5pt}
	\caption[Diagram of the logical pipes within the Bluetooth USB transport.]{Diagram showing the logical USB pipes within the Bluetooth USB physical transport.}
	\label{fig:btusbtransport}
\end{figure}

Data and event packets received from the USB Bluetooth adapter's logical pipes are written into the packet buffer indicated in the stack instance configuration, and dispatched to the stack via the \lstinline{Bluetooth_ProcessPacket()} function. There they are processed internally via the Bluetooth stack, moving up the HCI, L2CAP and user-provided service layers as appropriate. The matching \lstinline{CALLBACK_Bluetooth_SendPacket()} callback function from the stack implements the data and HCI control packet transmission code back to the adapter, sourcing the generated packets from the same packet buffer location.

\FloatBarrier
\subsection{HCI Layer}

Sitting at the lowest level of the core stack, the HCI layer performs the main role of managing connections between Bluetooth devices, as well as processing events from the Bluetooth controller, and data encapsulation/decapsulation for transport over the Bluetooth link. To perform these tasks, the HCI layer is invoked each time a data packet is received or ready to be sent, an event is received from the controller, or during idle periods of the user application for general connection management.

The current HCI layer state is implemented as a state machine, which is automatically advanced through the appropriate states on stack initialization to configure the attached Bluetooth device, according to the configuration parameters set in the stack instance configuration structure. These states, shown in Listing \ref{lst:hcistates}, correspond to the asynchronous initialization steps performed by the stack when the HCI layer is reset.

\lstinputlisting[float=tbph,caption={State machine states for the HCI management layer.},label={lst:hcistates}]{./Figures/HCIStates.c}

The HCI baseband controller management in the HCI layer is split into two distinct parts; an event processing routine to process commands and manage transitions between HCI states, and the state machine processing routine to execute commands based on the current HCI state. As commands must only be sent once on the transition between HCI state machine states, a flag \texttt{StateTransition} in the HCI state structure of each Bluetooth stack instance indicates if a state transition has occured within the event processing function, so that the state machine processor can issue the appropriate responses on transisiton edges only.

State timeouts within each of the HCI states is managed via a \texttt{TicksElapsed} counter in the HCI state structure of each Bluetooth stack instance. This counter, incremented each time the HCI layer is notified by the user application of a tick period elapsing, forces a resend of the current HCI state's command by forcing the state transition flag \texttt{StateTransition} high when the timeout period has expired. This system ensures that HCI commands dropped during the initialisation of the module are properly re-issued.

If a connection is requested by a remote device, the HCI layer will issue a \\ \lstinline{CALLBACK_Bluetooth_ConnectionRequest()} callback into the user application, if space is available within the HCI connection object pool. If no space is available (i.e. all HCI channel objects are currently allocated for established connections) the connection request is silently discarded with the Bluetooth error code \lstinline{HCI_ERROR_LIMITED_RESOURCES}. If the user application accepts the connection request, the stack will issue an acceptance response to the remote device. If rejected, a suitable rejection notification is issued by the stack's HCI layer.

Authentication is currently implemented in a slave connection role only, as a PIN based authentication. All sent connection requests and acceptance packets indicate that the device requires a slave role in the connection establishment, indicating that the remote device is responsible for requesting the authentication PIN code from the user, which the local stack then authenticates against the PIN code configured in each stack instance.

Inside the HCI layer, received data packets are stripped of their HCI layer headers and passed upwards to the L2CAP layer for further processing. Packets sent from the L2CAP layer to the HCI layer undergo the opposite process; a HCI layer packet header is prefixed onto the packet contents and passed to the user application for transmission to the Bluetooth controller.

\FloatBarrier
\subsection{L2CAP Layer}

% TODO

\FloatBarrier
\subsection{Bluetooth Services}

% TODO

\FloatBarrier
\subsubsection{SDP Service}

% TODO

\FloatBarrier
\subsubsection{HID Service}

% TODO

\FloatBarrier
\subsubsection{RFCOMM Service}

% TODO

\section{Integration into User Applications}

In order to implement the Bluetooth stack into a user application, the entire base stack (excluding the services) must be added to the project for compilation. The user configuration and interface to the stack is made via the header file \texttt{BluetoothCommon.h}. User applications attempting to use the stack must be compiled using the C99 language specification, and a GCC compiler derivative.

\FloatBarrier
\subsection{Management Functions}

The following functions must be executed by the user application to operate the stack correctly.

\vspace{1em}
\begin{lstlisting}
	void Bluetooth_Init(BT_StackConfig_t* const StackState);
\end{lstlisting}

The stack initialization function must be called once on system startup, to initialize a given instance of the stack ready for use.

\vspace{1em}
\begin{lstlisting}
	bool Bluetooth_ManageConnections(BT_StackConfig_t* const StackState);
\end{lstlisting}

This function must be executed periodically to manage existing connections and reply to any pending L2CAP or HCI events. The exact frequency this function must be called at is not a concrete value, rather, it should be called as fast as the user application will allow. Failure to call this function fast enough will introduce latency when establishing and maintaining new connections and logical channels or, in extreme cases, will cause connection and channel negotiations to fail.
	
\vspace{1em}
\begin{lstlisting}
	void Bluetooth_ProcessPacket(BT_StackConfig_t* const StackState,
	                             const uint8_t Type);
\end{lstlisting}

When a new packet is received from the Bluetooth controller, this function must be invoked. Note that the received packet data should be stored into the packet buffer set in the Bluetooth stack instance's configuration structure prior to invoking the packet processing routine.

\vspace{1em}
\begin{lstlisting}
	void Bluetooth_TickElapsed(BT_StackConfig_t* const StackState);
\end{lstlisting}

To manage timeouts, the stack must be updated reguarly of the passage of time using this function. Each time the configured \lstinline{BT_TICK_MS} tick period elapses, this function should be invoked to determine if any timeout conditions must be generated internally.

\FloatBarrier
\subsection{Events and Callbacks}

In response to conditions set within the stack internals, one or more event or callback functions may be invoked. These functions, implemented by the user application, allow the system to respond to events in a synchronous manner as they occur within the stack, and/or indicate to the stack the appropriate action to take when a decision is required. The user-implemented functions are split into two categories:

\begin{enumerate}
	\item \textbf{Event Functions}, which allow the user application to respond to events. This class of function has no return value, and the user application may choose to not provide a functional implementation for one or more events, if the event condition is of no interest to the user application.
	\item \textbf{Callback Functions}, invoked when the stack reaches a condition where a decision must be made by the user application---such as whether to accept or reject a particular incoming connection or channel request---before stack operations can continue. These functions require a decision to be give as the function's return value, which is then used by the stack internally.
\end{enumerate}

Event and callback invocations are generated by the stack layers as required. Figure \ref{fig:messageresponsecallback} shows the general message, callback and response sequence path a typical software stack will follow throughout its normal processing activities.

\begin{figure}[tbph]
	\vspace{1em}
	\centering
		\includegraphics[width=140mm]{MessageResponseCallback.png}
	\rule{35em}{0.5pt}
	\caption[Diagram of the Message, Response and Callback sequence]{Diagram showing the Message, Response and Callback sequence of a typical software stack when passing a message and response through several intermediate layers.}
	\label{fig:messageresponsecallback}
\end{figure}

In the project's Bluetooth stack, the names of the event and callback functions are fixed, and cannot be changed. User applications are required to implement all callback and event functions (identified via their \lstinline{CALLBACK_} and \lstinline{EVENT_} prefixes) using the exact prototypes given in the module headers. This design was preferred over a traditional registration system to reduce the stack's binary size and RAM footprint.

A minimal set of callback and event functions (excluding the physical packet transport callback) required to link in the SDP and RFCOMM services to the core Bluetooth stack are shown in Listing \ref{lst:minbteventcallbacks}.

\lstinputlisting[float=tbph,caption={Minimal set of event callbacks for SDP and RFCOMM service use.},label={lst:minbteventcallbacks}]{./Figures/MinimalBTEventCallbacks.c}

