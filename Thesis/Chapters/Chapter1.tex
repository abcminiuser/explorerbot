\chapter{Overview}
\label{Chapter1}
\lhead{Chapter 1. \emph{Overview}}

In almost all modern portable consumer devices, Bluetooth plays a large role; it is available in
the vast majority of mobile phones and their associated accessories, in cars, in laptops and, most
recently, in mobile tablet PCs. Bluetooth as a technology gives a standardized low power wireless
communications standard from the baseband up to the higher level services, allowing implementing
devices to communicate with one another in a manufacturer-agnostic way. This freeing of consumers
from the proprietary short range wireless solutions (such as \emph{ZigBee}) has helped make Bluetooth
the wireless communication system of choice for many applications.

\section{Project Background}

Despite this ubiquity, Bluetooth remains firmly in the realm of systems containing large amounts of
RAM, storage, processing power and - in many cases - full operating system stacks. For small-scale
embedded devices with tiny 8-bit processors, clock speeds in the tens of MHz (or even less) and RAM
measured in kilobytes, Bluetooth remains impractical either due to its expense or the lack of suitable
software.

However, existing solutions do exist. System designers can integrate off-the-shelf Bluetooth solutions
in their products; small hardware modules containing the Bluetooth baseband and a fixed-function
microprocessor, which handles the complex onion-like layers of the various Bluetooth stack components.
These modules are generally fixed function however, making them unsuitable in applications where a specific 
or even custom Bluetooth service is required. In addition, such modules are generally significantly more
expensive than the product's main processor, negating its cost/benefit ratio where a more powerful
system processor could be substituted to manage the entire application including the Bluetooth component.

These turn-key modules are made all the less attractive when one considers the cost of a raw Bluetooth
baseband IC module, without an integrated processor to manage the software stack; these are generally
available from multiple vendors at costs measured in the sub-US\$5 range. This indicates that the main cost
of the complete modular solutions lies not in the physical hardware, but the IP of the Bluetooth software
stack. If such a stack could be made widely available for use in embedded systems, this fixed-cost vendor
lock-in could be avoided and cheaper Bluetooth enabled systems developed for both hobbyist and commercial
use.

\section{Project Brief}

% TODO

% Explain project - embedded stack and robot

\section{Design Goals}

For such a stack to be useful in an embedded environment, it must be able to conform
to the restrictions such an environment imposes. Specfically, the completed stack must
minimize its compiled and working set footprints, reduce or eliminate the need for
dynamic memory allocation, and minimize its hardware dependencies to suit as wide a
range of processors of differing capabilities as possible.

The design goals of the complete stack were therefore set to:

\begin{itemize}
	\item Use as little RAM as possible
	\item Compile to as small a binary as is practical
	\item Offer a framework upon which services can be added to suit a particular application
	\item Provide asynchronous events to which the user application can respond to
	\item Allow for a variable number of simultaneous connections and logical channels to/from remote devices
	\item Have no requirement for dynamic memory allocation on the heap
	\item Be fully decoupled from the physical transport to the Bluetooth Adapter
	\item Be endian-correct regardless of native processor endianness
\end{itemize}

% TODO